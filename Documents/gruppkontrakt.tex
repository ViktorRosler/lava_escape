\documentclass{mall}
\usepackage[utf8]{inputenc}

\newcommand{\version}{Version 1.1}
\author{Daniel Huber, \url{danhu849@student.liu.se}\\
  Viktor Rösler, \url{vikro653@student.liu.se}\\
  }
\title{Gruppkontrakt:\\ Daniel Huber och Viktor Rösler}
\date{2020-11-14}
\rhead{Daniel Huber, danhu849 \\
Viktor Rösler, vikro653}


\begin{document}
\projectpage
\tableofcontents

\newpage

\section{Revisionshistorik}
\begin{table}[!h]
\begin{tabularx}{\linewidth}{|l|X|l|}
\hline
Ver. & Revisionsbeskrivning & Datum \\\hline
1.1 & Gruppkontrakt Daniel Huber och Viktor Rösler & 201114 \\\hline
1.0 & Gruppkontrakt -- Exempel & 201001 \\\hline
\end{tabularx}
\end{table}

\section{Projektmedlemmars önskemål och farhågor}
\label{prereq}

\begin{itemize}
\item \textbf{Daniel Hubers önskelista för optimerad egeninsats i grupparbete:}

  \begin{itemize}
  \item Förväntar mig att gruppmedlemmar säger till om något saknas för att optimera deras arbetsflöde.
  \item Min dygnsrytm är från mellan 4 och 5 till 19 och 20. Effektiv jobbtid med lunch mellan 06 och 18.
  \item Får absolut inte bli störd när jag sover.
  \item Måste få tillfälle att motionera 1-1.5 timmar per dag.
  \item Jobbar bäst i en miljö där jag bara hör min datorfläkt och mina knapptryckningar på tangentbordet.
  \item Måste följa mitt mat- och sovschema för att inte riskera att bli trött och grinig.
  \item Arbetar gärna på helger om gruppmedlem gör detsamma.
  \item Medan jag arbetar behöver jag ta små pauser ungefär varannan timme.
  \item Måste ha goda marginaler till deadline och behöver en planering.
  \item Är bra på att planera, skriva planeringen och framförallt följa den.
  \item Förväntar mig att gruppen också följer planeringen, men meddelar exakt då något försenas.
  \item När jag känner att någon tagit för lång tid med ett arbete och inte kommer hinna till deadline utan att säga till tenderar jag att ta över uppgiften och köra över alla andras åsikter bara för att möta deadline.
  \item Inte rädd att lägga ner mycket tid på arbetet, men förväntar mig då samma av gruppmedlemmar.
  \item Vill ha kritik och feedback direkt.
  \item Förväntar mig att bli konfronterad vid oklarheter under arbetetsgång gällande arbetet jag utför eller planeringen.
  \item Blir mer driven av komplimanger om mitt utförda arbete.
  \item Jag tycker inte om att inte få kredibilitet när andra använder mitt arbete.
  \item Behöver ha någon form av möte/kontroll/snack om arbetets progression sen dagen innan och vad som planerats för kommande dag. 
  \item Vill att det jag producerar ska hålla hög kvalitet, men inser vikten av att först uppfylla de grundläggande kraven och får inte ångest över att sänka mina förväntningar för att hinna lämna in arbete innan deadline. 
  \item Tycker det inte är värst kul att skriva dokument, men jag är bra på det.
  \item Frågar alltid gruppmedlemmar om hjälp när jag inte vet något och känner att jag inte kan ta reda på det själv inom rimlig tid.
  \item Svarar gärna på frågor och förklarar gärna programmeringskoncept.
  \item Är bra på att snabbt ta in information från stora texter och tycker jag har lätt för att sortera även större informationskällor.
  \item Nås lättast via telefon: 072-565 35 22 eller Discord.
  \end{itemize}

\item \textbf{Viktor Röslers önskelista för optimerad egeninsats i grupparbete:}
  \begin{itemize}
  \item Min dygnsrytm är från mellan 7 och 8 till 23 och 00. Jag behöver egentid från 17 till 19 för att motionera och äta.  
  \item Jag tycker det är viktigt att man kan ta ledigt på kvällar och helger om man vill, men jag planerar att arbeta på projektet även då.  
  \item Mitt främsta mål med projektet är att bli bättre på mjukvaruutveckling. Mål om högre betyg anpassar jag efter gruppens önskemål. 
  \item Jag kommer att vara ute i god tid med att utföra arbetsuppgifter och möta dealines. Om andra gruppmedlemmar vill vänta kommer jag att arbeta i förväg på egen hand.
  \item Behöver ha någon form av frekventa möten/kontroller/snack om arbetets progression och vad som ska göras näst.
  \item Jag vill att arbetsuppgifter som ingen gruppmeddlem tycker är så värst kul delas upp, och att arbetsuppgifter vi tycker är kul görs tillsammans. 
  \item Det är viktigt för mig att alla gruppmedlemmar blir hörda och får möjlighet att påverka.
  \item Gillar att lägga extra tid på programmeringen för att höja kvaliteten på produkten.   
  \item Är villig att hjälpa andra gruppmedlemmar med att felsöka deras kod.
  \item Diskuterar gärna alternativa lösningar, och är öppen för förslag om hur projektet utförs på bästa sätt.
  \item Om jag anser att det finns ett bättre sätt att göra något på kommer jag att säga det, och jag önskar att andra gruppmedlemmar förmedlar liknande kritik till mig.
  \item Nås lättast via skriftlig kommunikation över Discord.\\
  \end{itemize}

  \newpage

\item \textbf{Detta oroar jag mig inför i det kommande projektet. Detta vill jag inte erfara igen.}
  \\\emph{ Projekt utförs sällan helt felfria eller utan problem som måste lösas. Ibland uppstår problem på gruppnivå som inte framkommer förrän i utvärderingen av ett projekts slutförande. Det kan vara känslan av utanförskap eller otillräcklighet. Att man tvingats utföra hela arbetet själv. Det viktiga är att blottgöra negativa erfarenheter projektmedlemmar haft i andra projekt för att kunna undvika eller förebygga dem i det kommande projektet alla redan från start. }

\item Daniel:
  \begin{itemize}
  \item Oroar mig för att inte hinna lära mig all c++ som behövs i projektet.
  \item Oroar mig för att inte kunna skriva tillräckligt bra c++ kod.
  \item Vill inte tvingas lämna in saker efter deadline eller hamna efter i planeringen igen.
  \item Vill inte erfara att skriva dokumenten helt själv igen.
  \end{itemize}


\item Viktor:
  \begin{itemize}
  \item Oroar mig för att skillnader i ambitionsnivå eller programmeringsförmåga inom gruppen kommer påverka resultatet negativt.
  \item Vill inte göra en planering av projektet som senare visar sig inte vara genomförbar.
  \item Vill inte lägga mycket tid på att utöka projektet under implementeringsfasen med saker som inte behövs igen.
  \end{itemize}
  
\end{itemize}

\section{Projektgruppens optimerade arbetsflöde}

\begin{itemize}
\item \textbf{Vilka tider arbetar vi, och vilka tider är vi nåbara utöver detta?}

  Vi har kommit överens om att arbeta vardagar mellan vaken tid och läggdags. Då vi har olika dygnsrytmer skickar vi ett discord meddelande när vi börjar jobba om vad vi jobbar med för att undvika dubbelarbete. En gång varje dag har vi ett möte där det diskuterats vad vi gjort och vad som kommer jobbas med de kommande 24 timmarna.

  De dagar det är skolaktiviteter träffas vi en kort stund efter skolaktiviteten för ett snabbt möte. Vissa saker förmedlas bäst ansikte mot ansikte än över discord.

\item \textbf{Hur kommunicerar vi med varandra? Vilka verktyg/kanaler använder vi? Hur och när är det okej att vi avbryter varandra?}

  Kommunikation förs huvudsakligen huvudsakligen över discord. Meddelanden besvaras när det finns tid, på så vis väljs det av en själv när avbrott i arbetet passar. Detta för att flow optimeras och förlängs så långt som möjligt. Det har bestämts att det alltid finns en annan åtagen uppgift att jobba med ifall en uppgift visas alltför svår att lösa på egen hand och det tar lång tid att få svar.  Medlemmarna i gruppen anses vara självständiga och driftiga och det litas på att den andre nyttjat de vanliga felsökningsvägarna innan förfrågan om hjälp skickas. När ett problem lösts meddelas det till den andre via discord.


  \item \textbf{Hur gör vi för att ge varandra möjlighet att framföra åsikter och tankar om uppgifter och idéer till arbetet?}

Enligt önskemålen ovan meddelas kritik och feedback direkt när något felaktigt eller konstigt upptäcks i kod eller text. Diskussioner gällande ens arbetsinsatser anses välkomna och det finns inga problem att kod och text skrivs om eller raderas om slutprodukten förbättras.\\

\item \textbf{Arbetar vi tillsammans med uppgifter, eller var för sig?}

  Daniel vill ha en femma i projektet och Viktor har gått med på att matcha det. Daniel anser sig bra på att skriva dokument då samtliga dokument i förra projektkursen fick betyget VG och har tagit sig an att ansvara för kravspecifikationen. Efter att kravspecifikationen fått ett VG delas resten av dokumenten upp vid ett senare tillfälle då vi bättre vet vad dokumenten ska innehålla samt vilka av de delar vi känner oss bäst att skriva.


\item \textbf{Hur fördelar vi ansvarsområden?}
  
  Viktors huvudsakliga mål är att bli bättre på att koda och är väldigt duktig på det. För att optimera chansen om en femma i kursen bör han få koda så mycket som möjligt. Viktor har översiktligt ansvar för koden och ser till att den är rättad innan inlämning medan Daniel har översiktligt ansvar för dokumenten. Daniel har fått uppgiften att hålla reda på allt som behöver göras i projektet, ha koll på hur gruppen ligger till i jämfört med planeringen och ser till att saker är klara i god tid innan deadline. 

\end{itemize}

\section{Risker och förebyggande åtgärder gällande orosmoment}

\begin{itemize}

\item \textbf{Orosmoment: För stor skillnad i ambitions- och programmeringsnivå mellan gruppmedlemmar}\\
Samtliga projektmedlemmar har samma strävan. Att dels få en femma och dels skapa ett så bra spel som möjligt. En slutprodukt att känna sig stolt över. Detta uppnås genom god beredskap och gott kontinuerligt samarbete. I detta dokument har projektmedlemmarnas styrkor och svagheter blottlagts. Daniel är inte lika bra som Viktor på att programmera, men är väldigt bra på att skriva dokument i LATEX. Därav har vissa projektuppgifter bestämts i förväg för att arbetsflödet ska optimeras och flaskhalsar minimeras. Individen ges möjlighet att bidra till projektet med sina bästa färdigheter. 


\item \textbf{Orosmoment: Försening i planering}\\
  Målet som projektgruppen ämnas att uppnå är högt, men inte onåbart. Kraven har diskuterats fram med kursledning för att få en femma. Finns det överambitiösa bör-krav riskeras utmattning och panikbeteende vid potentiell försening. Detta måste undvikas. Planeringen har satts upp till att alltid vara en vecka före kursens planering. På så vis har en tidsbuffert förberetts för oförutsedda tidssänkor. Förseningar i tidigare projekt har erfarits av samtliga projektmedlemmar och dess negativa konsekvenser avskys av båda. Därav finns ett egenvärde i att projektplaneringen åtföljs.


\item \textbf{Risk: Sjukdom \& Corona}\\
  Det ska understrykas att den personliga hälsan för projektets medlemmar alltid kommer först. Med stora åtaganden krävs en ödmjukhet inför det oförutsedda och respekt för dess konsekvenser om de inte tas på allvar. Mildare sjukdomssymptom eller annan vag arbetsnedsättning förväntas meddelas till projektgruppen vid första antydan. Det avgörs av den enskilde själv vilken arbetsbörda denne kan åta sig, men första prioritet ligger i att bli frisk och komma tillbaka till arbetet med full kraft. Därav uppmuntras den berörde att använda en dag eller två till lätt arbete eller vila.

  Vid grövre sjukdom, eskalering av en mildare sjukdom eller olycka av allvarligare sort meddelas projektgruppen omedelbart och kontakt med kursledning tas omgående. Samtidigt aktiveras katastrofprotokollet enligt nedan:
  \begin{enumerate}
  \item Den drabbade lägger sitt eget tillfrisknande som första och enda prioritet.
  \item Projektgruppen flyttar fullt fokus att uppfylla alla ska-krav i projektet.
  \item Samtliga dokument ska bara uppfylla minsta möjliga för godkänt.
  \item Alla bör-krav läggs åt sidan tills dess att ska-kraven och dokumenten är färdiggjorda.
  \item Förväntningar om högre betyg än 3 slopas tills dess att ska-kraven och dokumenten är klara.
  \item Samtliga projektmedlemmars skuldbeläggning, frustration och offermentalitet skjuts åt sidan eller glöms bort.
  \end{enumerate}
  

\end{itemize}

\section{Utvärdering}

\begin{itemize}
\item \textbf{När ska vi påminna oss om gruppkontraktet och utvärdera hur det fungerat?}

  \emph{Insikt: Gruppkontraktet ska vara ett stöd för arbetet i kursen/projektet. Om det finns saker som
    inte fungerar i gruppen behöver det kanske omarbetas. Det kanske finns problem med att det existerande
    kontraktet inte efterlevs av alla deltagande. Kontraktet kan aldrig vara heltäckande och måste stödjas
    av en vilja att ha ett gott samarbete inom gruppen}
  
  Kontraktet verkar som garanti för projektmedlemmars lov till insats. Vid grov avvärjning från kontraktets delar eller dess helhet kontaktas i första hand den avvärjande projektmedlemmen och i andra hand kursledningen. Diskussion gällande kontraktets innehåll är alltid välkommet att diskuteras, men det förutsätts att detta sker innan eventuellt kontraktsbrott. Utvärdering gällande kontraktets utformning planeras hållas i mitten av projektets gång efter första veckan av kodning (v.49). Därutöver hålls en slutgiltig utvärdering efter eller nära projektets slut gällande hela projektets genomförande.

  %% Vid utsatt tid: utvärdera hur gruppkontraktet har följts, fundera på ifall något i kontraktet
  %% behöver ändras, eller om något nytt behöver läggas till.

\end{itemize}

\section{Underskrifter}

\vspace{2cm}
\noindent\begin{tabular}{ll}
\makebox[2.5in]{\hrulefill} & \makebox[2.5in]{\hrulefill}\\
Daniel Huber & Datum\\[16ex]% adds space between the two sets of signatures
\makebox[2.5in]{\hrulefill} & \makebox[2.5in]{\hrulefill}\\
Viktor Rösler & Datum\\
\end{tabular}

\end{document}
